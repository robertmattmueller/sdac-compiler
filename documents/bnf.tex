\documentclass[letterpaper]{article}

\usepackage{listings}
\lstset{
    basicstyle=\ttfamily,
    mathescape
}


\begin{document}
\section*{BNF clauses for state-dependent action costs}
Most of the syntax is already provided by the PDDL 3.1 language. Only \texttt{sum} and
\texttt{prod} are new language constructs.

\begin{lstlisting}

 <sdac-term>    ::= <arithm-term>
                  | <logical-term>

 <arithm-term>  ::= <number>
                  | (<binary-op> <sdac-term> <sdac-term>)
                  | (<multi-op> <sdac-term> <sdac-term>$^+$)
                  | (- <sdac-term>)
                  | (sum (<typed list(variable)>) <sdac-term>)
                  | (prod (<typed list(variable)>) <sdac-term>)

 <logical-term> ::= <atomic formula(term)>
                  | (not <logical-term>)
                  | (and <logical-term>$^*$)
                  | (or <logical-term>$^*$)
                  | (exists (<typed list(variable)>) <logical-term>)
                  | (forall (<typed list(variable)>) <logical-term>)


 <binary-op>    ::= - | /
 <multi-op>     ::= * | +

 <typed list (x)>    ::= x$^*$
 <typed list (x)>    ::=$^{:typing}$ x$^+$ - <type> <typed list(x)>

 <variable>          ::= ?<name>
 <atomic formula(t)> ::= (<predicate> t$^*$)
 <predicate>         ::= <name>
 <term>              ::= <name> | <variable>

 <name>         ::= <letter> <any char>$^*$
 <letter>       ::= a..z | A..Z
 <any char>     ::= <letter> | <digit> | - | _
 <number>       ::= <digit>$^+$ [<decimal>]
 <digit>        ::= 0..9
 <decimal>      ::= .<digit>$^+$


\end{lstlisting}

\end{document}
